\documentclass[11pt, a4paper]{article}
 
\begin{document}

Let $h$ be the list of the houses sorted in increassing order.
Let $l$ be the length of the list.

Whenever we go over an unvisited house, we loose nothing by visiting it, because it takes 0 time.
Due to this, at any point of the solution, we have the choice between going to the closest house negatively or positively.

Let us assume we are picking the $k$th house of out solution $s$ constisting of the indexes of picked houses.
Let's ignore our first house pick and assume $k \geq 1$.

Our current position is $h_{s_{k-1}}$.
Let $p_{k - 1}$ be the index of the next positive house available, and $q_{k - 1}$ be the next negative house available.


If we choose $h_{p_{k - 1}}$, then our new state would be:

\[p_k = p_{k - 1} + 1\]
\[q_k = q_{k - 1}\]
\[s_k = p_{k - 1}\]

The value of $p_k$ is based on the fact that houses in $h$ are sorted.

If we choose $h_{q_{k - 1}}$, then our new state would be:

\[p_k = p_{k - 1}\]
\[q_k = q_{k - 1} - 1\]
\[s_k = q_{k - 1}\]

The value of $q_k$ is based on the fact that houses in $h$ are sorted.


We can calculate the cost of our choice as:

\[|h_{s_{k}} - h_{s_{k - 1}}| \times (l - k - 1) + c_{p_k, q_k, s_k}\]

with $c_{p_k, q_k, s_k}$ being a cost we have to calculate based on $(p_k, q_k, s_k)$. We make our choice based on this value, but since it is recursivly defined we need a base case which is when $k = l - 1$, where the cost is simply $0$.

Based on both above possibillities, we can deduce the following:

\[\forall k \in [1, l[, s_k = q_{k - 1} = q_k + 1 \vee s_k = p_{k - 1} = p_k - 1\]

We know that space to explore is defined by (p, q, s).
Since \[\forall k \in [1, l[ p_k \in [0, l[, q_k \in [0, l[, s_k = q_k + 1 \vee s_k = p_k - 1\] we have to explore a space of size $l * l * 2$.
By using memoisation, we can explore each state only once.
This gives a complexity of $2n^2$.
We still have to handle the choice of $s_0$.
This is rather trivial, we just need to calculate the cost of starting with the smallest positive house or smallest negative house.
Since the cost is linear, it doesn't affect the complexity of our algorithm.


Note: Our memoization implementation uses the python dictionary type. We can reasonably expect this dictionnary type to have a polytnomial time or better, and thus our algorithm remains polynomial. It is most likelly constant or logaritmic in complexity. Our result is created using a python list. Appending and lookup should reasonably be polynomial too (and is likelly constant or linear).


\end{document}